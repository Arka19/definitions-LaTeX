% !TEX root = ./main.tex

\section{Computational Indistinguishability}

\begin{definition}[Computational Indistinguishability]
	\index{Indistinguishability!Computational}
	Two ensembles $X=\{X_\alpha\}_{\alpha \in S}$ and $Y=\{Y_\alpha\}_{\alpha \in S}$ are said to be computationally indistinguishable, denoted by $X \approx_c Y$, if for every non-uniform $\ppt$ distinguisher $\ddv$, every polynomial $\pp$, all sufficiently large $\secpar$ and every $\alpha \in \bin^{\poly} \cap S$
	\[
		\Big| \prob{\ddv(\secparam,X_\alpha)=1} - \prob{\ddv(\secparam,Y_\alpha)=1} \Big| < \frac{1}{\pp(\secpar)}\enspace,
	\]
	where the probability are taken over the samples of $X_\alpha$, $Y_\alpha$ and coin tosses of $\ddv$.
\end{definition}


\section{Statistical Indistinguishability}

\begin{definition}[Statistical Indistinguishability]
	\index{Indistinguishability!Statistical}
	Two ensembles $X=\{X_\alpha\}_{\alpha \in S}$ and $Y=\{Y_\alpha\}_{\alpha \in S}$ are said to be statistically indistinguishable, denoted by ${X \approx_s Y}$, if for every polynomial $\pp$, all sufficiently large $\secpar$ and every $\alpha \in \bin^{\poly} \cap S$
	\[ 
	 \Delta(X_\alpha,Y_\alpha) < \frac{1}{\pp(\secpar)}\enspace,
	\] 
	where $\Delta(X_\alpha,Y_\alpha)$ corresponds to the statistical distance between $X_\alpha$ and $Y_\alpha$.
\end{definition}

