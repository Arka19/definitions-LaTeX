\documentclass{book}
\usepackage{fullpage}


\usepackage{charter}
%\usepackage{kpfonts}
%\usepackage{times}

\usepackage{geometry}
\usepackage[dvipsnames]{xcolor}
\usepackage{amsmath}



\usepackage{amsthm}
\makeatletter
\def\th@plain{%
	\thm@notefont{}% same as heading font
	\itshape % body font
}
\def\th@definition{%
	\thm@notefont{}% same as heading font
	\normalfont % body font
}
\makeatother
%The above is to make the theorem titles appear in bold

\let\circledS\relax
\usepackage{amssymb}
\usepackage{amsfonts}
\usepackage{mathtools}
\usepackage{xspace}
\usepackage{bm}
\usepackage{enumitem}

% \usepackage[most]{tcolorbox}

% \newtcolorbox{defbox}[1][]
% {
% 	sharp corners,
% 	colback=cyan!7!white,
% 	breakable,
% 	enhanced,
% 	#1,
% }


% \newtcolorbox{thmbox}[1][]
% {
% 	sharp corners,
% 	colback=green!30!white,
% 	breakable,
% 	enhanced,
% 	#1,
% }

% \newtcolorbox{notebox}[1][]
% {
% 	sharp corners,
% 	colback=red!5!white,
% 	breakable,
% 	enhanced,
% 	#1,
% }



\newtheorem{proposition}{Proposition}
\newtheorem{theorem}{Theorem}
\newtheorem{definition}{Definition}
\newtheorem{lemma}{Lemma}
\newtheorem{claim}{Claim}
\newtheorem{remark}{Remark}
\newtheorem{corollary}{Corollary}
\newtheorem{observation}{Observation}
\newtheorem{example}{Example}

\usepackage{boxedminipage}
\usepackage{xparse}
\usepackage{float}
\usepackage{caption}
\usepackage{ifthen}

\usepackage{imakeidx}
\makeindex







%\usepackage{ulem}
\usepackage[
lambda,
operators,
advantage,
sets,
adversary,
landau,
probability,
notions,	
logic,
ff,
mm,
primitives,
events,
complexity,
asymptotics,
keys]{cryptocode}

%comment commands
\newcommand{\arka}[1]{{\color{brown} Arka: #1}}


\renewcommand\labelitemi{--}

%\usepackage{geometry}


\usepackage[pdftex,bookmarks=true,pdfstartview=FitH,colorlinks,linkcolor=Cerulean,filecolor=Cerulean,citecolor=NavyBlue,urlcolor=Cerulean]{hyperref}
\pagestyle{plain}




% !TEX root = ./main.tex



\newcommand{\bfmatrix}[1]{\mathbf{#1}}
\renewcommand\vec{\mathbf}
\newcommand{\llbrace}{\left\lbrace}
\newcommand{\rrbrace}{\right\rbrace}
\newcommand{\round}[1]{\left\lfloor #1 \right\rceil}


\newcommand{\PR}{\mathbb{P}}
\newcommand\inner[2]{\langle #1, #2 \rangle}
\newcommand{\randomOracle}{\mathcal{O}}
\newcommand{\oracle}[1]{\mathcal{#1}}
\newcommand{\pspace}{\mathsf{PSPACE}}
\newcommand{\iproof}{\mathsf{IP}}
\newcommand{\CRHF}{\mathsf{CRHF}}
\newcommand{\OWP}{\mathsf{OWP}}
\newcommand{\OWF}{\mathsf{OWF}}
\newcommand{\TDF}{\mathsf{TDF}}
\newcommand{\ckt}{\mathsf{C}}
\newcommand{\alice}{\mathsf{Alice}}
\newcommand{\bob}{\mathsf{Bob}}
\newcommand{\eve}{\mathsf{Eve}}
\newcommand{\UOWHF}{\mathsf{UOWHF}}
\newcommand{\dCRHF}{\mathsf{dCRHF}}
\newcommand{\DTIME}{\mathsf{DTIME}}
\newcommand{\DSPACE}{\mathsf{DSPACE}}
\newcommand{\BPTIME}{\mathsf{BPTIME}}
\newcommand{\SZK}{\mathsf{SZK}}
\newcommand{\primitive}[1]{\mathsf{#1}}
\newcommand{\machine}[1]{\mathsf{#1}}
\newcommand{\lang}{\mathcal{L}}

%ciphertext
\newcommand{\ciphertext}{\mathsf{ct}}

%Commitment
\newcommand{\com}{\mathsf{com}}

%Obfuscation
\newcommand{\iOM}{\mathsf{iOM}}

%Witness Encryption
\newcommand{\WE}{\mathsf{WE}}

\newcommand{\NIWI}{\mathsf{NIWI}}

\newcommand{\prove}{\mathsf{Prove}}
\renewcommand{\verify}{\mathsf{Verify}}

\newcommand{\view}{\mathsf{View}}
\newcommand{\out}{\mathsf{Out}}


\newcommand{\gen}{\mathsf{Gen}}


\begin{document}
	

\title{\LaTeX definitions for Cryptography Primitives}
\author{}
\date{}
 
\maketitle

Collect definitions of cryptographic primitives so that it is easy to copy and use while writing papers. I will try as far as possible to stick to the macros defined in cryptocode. 



\newpage 
\tableofcontents

\newpage

\chapter{Notation}
	% !TEX root = ./main.tex

We shall denote by $\out_A\langle A(a), B(b)\rangle$ the output of party $A$ on execution of the protocol between $A$ with input $a$, and $B$ with input $b$. By $\view_A\langle A(a), B(b)\rangle$, we denote the view of party $A$ consisting of the protocol transcript along with its random tape.

\chapter{Probability}
	% !TEX root = ./main.tex

\section{Computational Indistinguishability}

\begin{definition}[Computational Indistinguishability]
	Two ensembles $X=\{X_\alpha\}_{\alpha \in S}$ and $Y=\{Y_\alpha\}_{\alpha \in S}$ are said to be computationally indistinguishable, denoted by $X \approx_c Y$, if for every non-uniform $\ppt$ distinguisher $\ddv$, every polynomial $\pp$, all sufficiently large $\secpar$ and every $\alpha \in \bin^{\poly} \cap S$
	\[
		\Big| \prob{\ddv(\secparam,X_\alpha)=1} - \prob{\ddv(\secparam,Y_\alpha)=1} \Big| < \frac{1}{\pp(\secpar)}\enspace,
	\]
	where the probability are taken over the samples of $X_\alpha$, $Y_\alpha$ and coin tosses of $\ddv$.
\end{definition}


\section{Statistical Indistinguishability}

\begin{definition}[Statistical Indistinguishability]
	Two ensembles $X=\{X_\alpha\}_{\alpha \in S}$ and $Y=\{Y_\alpha\}_{\alpha \in S}$ are said to be statistically indistinguishable, denoted by ${X \approx_s Y}$, if for every polynomial $\pp$, all sufficiently large $\secpar$ and every $\alpha \in \bin^{\poly} \cap S$
	\[ 
	 \Delta(X_\alpha,Y_\alpha) < \frac{1}{\pp(\secpar)}\enspace,
	\] 
	where $\Delta(X_\alpha,Y_\alpha)$ corresponds to the statistical distance between $X_\alpha$ and $Y_\alpha$.
\end{definition}



\chapter{Symmetric Key Primitives}
	% !TEX root = ./main.tex

\section{One-way Function}

\begin{definition}[One-way Function]
		\index{One-way Function}
		\label{def:strongOWF}
		A function $f:\bin^{*} \mapsto \bin^{*}$ is a one way function if it satisfies the following two conditions: 
		\begin{enumerate}
			\item \emph{ Easy to compute}: There is a $\ppt$ algorithm $C$ s.t. $\forall x \in \{0, 1\}^{*}$,
			\[
				\prob{r \sample \bin^{m}~:~C(x;r) = f(x)} =1.
			\]
			\item \emph{Hard to invert}: For every non-uniform $\ppt$ adversary $\adv$, 
			\[
				\probsublong{x \sample \bin^\secpar, \widetilde{x} \leftarrow \adv(\secparam, f(x))}{f(\widetilde{x})=f(x)} \le \negl
			\]		
		\end{enumerate}
	\end{definition}


\section{Pseudorandom Generators}


\begin{definition}[Psedudorandom Generators]
	\index{Pseudorandom Generators (PRG)}
	A deterministic function $\prg: \bin^\secpar \rightarrow \bin^{p(\secpar)}$ is called a pseudorandom generator (PRG) if:
	\begin{enumerate}
		\item (efficiency): $\prg$ can be computed in polynomial time,
		\item (expansion): $p(\secpar) > \secpar$,
		\item $\llbrace x \sample \bin^\secpar: \prg(x) \rrbrace \approx_c \llbrace U_{p(\secpar)}\rrbrace$ , where $U_{p(\secpar)}$ is the uniform distribution over $p(\secpar)$ bits.
	\end{enumerate}
\end{definition}


\section{Non-interactive Commitment Schemes}
	
	We define below bit commitment schemes
	\begin{definition}[Non-interactive Bit Commitment Schemes]
		\index{Non-interactive Commitment}
		A polynomial time computable function: $\com: \bin \times \bin^\secpar \mapsto \bin^{\ell(\secpar)}$ is a bit commitment if it satisfies the properties below:
		\begin{description}
			\item[Binding:] For any $r,r' \in \bin^\secpar, b,b' \in \bin$, if $\com(b;r)=\com(b';r')$ then $b=b'$.
			\item[Computational Hiding:] The following holds:
			\[
				\Big\lbrace \com(0):r \sample \bin^\secpar \Big\rbrace \approx_c \Big\lbrace \com(1;r):r \sample \bin^\secpar \Big\rbrace \enspace.
			\]			
			where computational indistinguishability is with respect to arbitrary non-uniform $\ppt$ distinguisher.
		\end{description}
		
	\end{definition}


	\section{Signature Scheme}
		
	
	
	\begin{definition}
		\index{Signature Scheme}
		\label{def:signatures}
		An signature scheme consists of three polynomial-time algorithms $(\gen,\sign,\verify)$. 
		\begin{itemize}
			\item[--] $\gen$ is $\ppt$ algorithm that takes as input $\secparam$ and generates a key and verification key. $(\sk,\vk) \leftarrow \gen(\secparam)$.
			\item[--] $\sign$ is a $\ppt$ algorithm that computes the signature on a message $m$.  $\sigma \coloneqq \sign(\sk,m)$.
			\item[--] $\verify$ is a deterministic algorithm verifies the signature using the verification key. $\verify(\vk, m,\sigma)$ returns 0 or 1.
		\end{itemize}
		A  signature scheme that is existentially unforgeable against chosen message attacks if the following hold. 
			\begin{description}
				\item[Correctness] For every message $m \in \mathcal{M}$ (message space),
				\[
					\prob{\verify(\vk,m,\sigma) =1~:~(\vk,\sk) \leftarrow \gen(\secparam), \sigma \leftarrow\sign(\sk,m)} = 1
				\]
				\item[Security] For any $\ppt$ adversary $\adv$
				\[
						\probsublong{\begin{array}{l}
						(\vk,\sk) \leftarrow \gen(\secparam)\\
						(m,\sigma) \leftarrow \adv^{\sign(\sk,\cdot)}(\vk)
					\end{array}}{ \begin{array}{l}
								\adv \text{ did not query }m\\
								\verify(\vk,m,\sigma) =1
						\end{array} 
					} < \negl 
				\]
				where $\adv^{\sign(\sk,\cdot)}$ indicates that $\adv$ has access to an oracle that returns the signature on the queried message $m$.
			\end{description}

	\end{definition}

\chapter{Public Key Primitives}

\chapter{Proofs}
	% !TEX root = ./main.tex



\section{Non-interactive Witness Indistinguishability (NIWI)}
	\label{sec:defn:NIWI}
	
	\begin{definition} 
	\index{Non-interactive Witness Indistinguishability}
	A non-interactive witness-indistinguishable proof system $\NIWI=(\prove,\verify)$ for an $\npol$ relation $R_\lang$ consists of two polynomial-time algorithms:
		\begin{itemize}
			\item a probabilistic prover $\prove(x,w,\secparam)$ that given an instance $x$, witness $w$, and security parameter $\secparam$, produces a proof $\pi$.
			\item a deterministic verifier $\verify(x,\pi)$ that verifies the proof.
		\end{itemize}
		We make the following requirements:
		\begin{description}
			\item[Completeness] for every $\secpar \in \NN, (x,w) \in R_\lang$,
			\[
				\probsublong{\pi \leftarrow \prove(x,w,\secparam)}{\verify(x,\pi)=1} = 1
			\]
			\item[Soundness] for every $x \notin \lang$ and $\pi \in \bin^*$,
			\[
				\verify(x,\pi)=0\enspace.
			\]
			\item[Witness Indistinguishability] It holds that
			\[
				\Big\lbrace \prove(x,w_0,\secparam) \Big\rbrace_{\begin{subarray}{l}\secpar,x,\\w_0,w_1 \end{subarray}} \approx_c 				\Big\lbrace \prove(x,w_1,\secparam) \Big\rbrace_{\begin{subarray}{l}\secpar,x,\\w_0,w_1 \end{subarray}}\enspace,
			\]
where $\secpar \in \NN, x \in \bin^\secpar, w_0,w_1 \in R_\lang(x).$
		\end{description}
	\end{definition}


\section{Interactive Proof}

\begin{definition}
		\index{Interactive Proof}
		An interactive protocol $(\prover, \verifier)$ between a polynomial time prover $\prover$ and $\ppt$ verifier $\verifier$, for a language $\lang \in \npol$ is an interactive proof (resp. argument) if the following holds.
		\begin{description}
			\item[Completeness:] For every $x \in \lang$,
			\[
				\prob{\out_\verifier\langle \prover(x,w), \verifier(x)\rangle=1} =1\enspace.
			\]
			\item[Soundness (resp. computational soundness):] For any non-uniform (resp. $\ppt$) $\prover^*$, there exists a negligible function $\negl[\cdot ]$ such that for all $\secpar\in\NN$ and $ x \in \bin^{\secpar}\setminus \lang$,
			\[
				\prob{\out_\verifier\langle \prover^*, \verifier(x)\rangle=1} \le \negl[\secpar]\enspace.
			\]
		\end{description}
	\end{definition}


\section{Zero-Knowledge}

An interactive proof (resp. argument) $(\prover, \verifier)$ between a polynomial time prover $\prover$ and $\ppt$ verifier $\verifier$, for a language $\mathcal{L}$ is a zero knowledge proof (resp. argument) if the following holds.

\begin{definition}[\emph{GMR}\cite{STOC:GolMicRac85} Zero-knowledge]
	\index{Zero Knowledge!GMR}
	 An interactive proof (resp. argument) $(\prover, \verifier)$ between a polynomial time prover $\prover$ and $\ppt$ verifier $\verifier$, for a language $\mathcal{L}$ is a GMR zero knowledge proof (resp. argument) if the following holds. For every $\ppt$ verifier $\verifier^*$, there exists a $\ppt$ simulator $\simulator_{\verifier^*}$, such that 
	\[
	\Big\lbrace \view_{\verifier^{*}}\langle \prover(x,w), \verifier^{*(x)}\rangle \Big\rbrace_{\begin{subarray}{l}\secpar\in\NN,\\x\in \mathcal{L}\cap\bin^{\secpar},\\w \in R_\mathcal{L}(x)\end{subarray}} \approx_c \Big\lbrace \simulator_{\verifier^{*}}(x) \Big\rbrace_{\begin{subarray}{l}\secpar\in\NN,\\x\in \mathcal{L}\cap\bin^{\secpar},\\w \in R_\mathcal{L}(x)\end{subarray}}\enspace.
	\]		
\end{definition}


\begin{definition}[\emph{Auxiliary-input} Zero-knowledge]
	\index{Zero Knowledge!Auxiliary input}
	An interactive proof (resp. argument) $(\prover, \verifier)$ between a polynomial time prover $\prover$ and $\ppt$ verifier $\verifier$, for a language $\mathcal{L}$ is an auxiliary-input zero knowledge proof (resp. argument) if the following holds. For every $\ppt$ verifier $\verifier^*$, there exists a $\ppt$ simulator $\simulator_{\verifier^*}$, such that 
	\[
	\Big\lbrace \view_{\verifier^{*}}\langle \prover(x,w), \verifier^{*}(x,y)\rangle \Big\rbrace_{\begin{subarray}{l}\secpar\in\NN,\\x\in \mathcal{L}\cap\bin^{\secpar},\\w \in R_\mathcal{L}(x)\\y \in \bin^*\end{subarray}} \approx_c \Big\lbrace \simulator_{\verifier^{*}}(x,y,1^t) \Big\rbrace_{\begin{subarray}{l}\secpar\in\NN,\\x\in \mathcal{L}\cap\bin^{\secpar},\\w \in R_\mathcal{L}(x)\\y \in \bin^*\end{subarray}}\enspace.
	\]		
\end{definition}
	
	


\begin{definition}[\emph{Universal-simulation} Zero-knowledge]
	\index{Zero Knowledge!Universal Simulation}
	An interactive proof (resp. argument) $(\prover, \verifier)$ between a polynomial time prover $\prover$ and $\ppt$ verifier $\verifier$, for a language $\mathcal{L}$ is a universal-simulation zero knowledge proof (resp. argument) if the following holds. There exists a $\ppt$ simulator $\simulator$, such that for every $\ppt$ verifier $\verifier^*$ of running time at most $t(\secpar)$,
	\[
	\Big\lbrace \view_{\verifier^{*}}\langle \prover(x,w), \verifier^{*}(x)\rangle \Big\rbrace_{\begin{subarray}{l}\secpar\in\NN,\\x\in \mathcal{L}\cap\bin^{\secpar},\\w \in R_\mathcal{L}(x)\end{subarray}} \approx_c \Big\lbrace \simulator(\verifier^{*},1^t,x) \Big\rbrace_{\begin{subarray}{l}\secpar\in\NN,\\x\in \mathcal{L}\cap\bin^{\secpar},\\w \in R_\mathcal{L}(x)\end{subarray}}\enspace.
	\]			
\end{definition}

\begin{definition}[\emph{Black-box-simulation} Zero-knowledge]
	\index{Zero Knowledge!Black-box Simulation}
	An interactive proof (resp. argument) $(\prover, \verifier)$ between a polynomial time prover $\prover$ and $\ppt$ verifier $\verifier$, for a language $\mathcal{L}$ is a black-box-simulation zero knowledge proof (resp. argument) if the following holds. There exists a $\ppt$ simulator $\simulator$, such that for every $\ppt$ verifier $\verifier^*$,
	\[
	\Big\lbrace \view_{\verifier^{*}}\langle \prover(x,w), \verifier^{*}\rangle \Big\rbrace_{\begin{subarray}{l}\secpar\in\NN,\\x\in \mathcal{L}\cap\bin^{\secpar},\\w \in R_\mathcal{L}(x)\end{subarray}} \approx_c \Big\lbrace \simulator^{\verifier^{*}}(x) \Big\rbrace_{\begin{subarray}{l}\secpar\in\NN,\\x\in \mathcal{L}\cap\bin^{\secpar},\\w \in R_\mathcal{L}(x)\end{subarray}}\enspace.
	\]		
\end{definition}

\chapter{Secure Computation}

\chapter{Obfuscation}
	% !TEX root = ./main.tex

\section{Indistinguishability Obfuscator for Turing Machines}

\begin{definition}[Indistinguishability Obfuscator for Turing Machines]
		A succinct indistinguishability obfuscator for Turing machines  consists of a PPT machine $\iOM$ that works as follows:
		\begin{itemize}
			\item $\iOM$ takes as input the security parameter $\secparam$, the Turing machine $\TM$ to obfuscate, an input length $n$, and time bound $t$.
			\item $\iOM$ outputs a Turing machine $\widetilde{\TM}$ which is an obfuscation of $\TM$ corresponding to input length $n$ and time bound $t$. $\widetilde{\TM}$ takes as input $x \in \bin^n$.

		\end{itemize}
		The scheme should satisfy the following requirements:
		\begin{description}
			\item[Correctness] For all $\secpar \in \NN$, for all $\TM \in \mathcal{M}_\secpar$, for all inputs $x \in \bin^n$, time bounds $t'$ such that $t' \le t$, let $y$ be the output of $\TM(x)$ after at most $t$ steps, then
			\[
				\prob{\widetilde{\TM} \leftarrow \iOM(\secparam,1^n,1^{\log t},\TM)~:~\widetilde{\TM}(x)=y} = 1\enspace.
			\]
			\item[Security] It holds that
			\[
						\set{\iOM(\secparam,1^n,1^{\log t},\TM_0)}_{\begin{subarray}{l}\secpar,t,n,\\ \TM_0,\TM_1\end{subarray}} \approx_c
							\set{\iOM(\secparam,1^n,1^{\log t},\TM_1)}_{\begin{subarray}{l}\secpar,t,n,\\ \TM_0,\TM_1\end{subarray}} \enspace,
			\]
			where $\secpar\in \NN$, $n\leq t\leq 2^{\secpar}$, and $\TM_0, \TM_1$ are any pair of machines of the same size such that for any input $x\in\bin^n$ both halt after the same number of steps with the same output.


			\item[Efficiency and Succinctness] We require that the running time of $\iOM$ and the length of its output, namely the obfuscated machine $\widetilde{\TM}$, is $\poly[|\TM|,\log t,n,\secpar]$. We also require that the running time $\tilde t_x$ of $\widetilde{\TM}(x)$ is $\poly[t_x,|\TM|, n,\secpar]$, where $t_x$ is the running time of $\TM(x)$.
		\end{description}
	\end{definition}

\section{Witness Encryption}

\begin{definition}
		A witness encryption scheme for an $\npol$ language $\lang$, with corresponding witness relation $R_\lang$, consists of the following two polynomial-time algorithms:
		
		\begin{description}
			\item[Encryption.] The probabilistic algorithm $\WE.\enc(\secparam,x,m)$ takes as input a security parameter $\secparam$, a string $x \in \bin^*$, and a message $m \in \bin$. It outputs a ciphertext $\ciphertext$.
			\item[Decryption.] The algorithm $\WE.\dec(\ciphertext,w)$ takes as input a ciphertext $\ciphertext$, a string $w \in \bin^*$. It outputs either a message $m \in \bin$.
		\end{description}
		The above algorithms satisfy the following conditions:
		\begin{itemize}
			\item \textbf{Correctness.} For any security parameter $\secpar$, for any $m \in \bin$, and for any $(x,w) \in R_\lang$, we have that
			\[
				\prob{\ciphertext \leftarrow \WE.\enc(\secparam,x,m)~:~ \WE.\dec(\ciphertext,w)=m} = 1\enspace.
			\]
			\item \textbf{Security.} For any non-uniform \ppt adversary $\adv$, there exists a negligible function $\negl[\cdot]$ such that for any $\secpar \in \NN$, and any $x \notin \lang$, we have that
			\[
				\set{\WE.\enc(\secparam,x,0)}_{\secpar \in \NN, x\notin \lang} \approx_c \set{\WE.\enc(\secparam,x,1)}_{\secpar \in \NN, x\notin \lang}\enspace.
			\]
		\end{itemize}
	\end{definition}







\newpage
\bibliographystyle{alpha}
\bibliography{./cryptobib/abbrev0,./cryptobib/crypto}

\newpage 
\printindex 



\end{document}
