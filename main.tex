\documentclass{book}
\usepackage{fullpage}


\usepackage{charter}
%\usepackage{kpfonts}
%\usepackage{times}

\usepackage{geometry}
\usepackage[dvipsnames]{xcolor}
\usepackage{amsmath}



\usepackage{amsthm}
\makeatletter
\def\th@plain{%
	\thm@notefont{}% same as heading font
	\itshape % body font
}
\def\th@definition{%
	\thm@notefont{}% same as heading font
	\normalfont % body font
}
\makeatother
%The above is to make the theorem titles appear in bold

\let\circledS\relax
\usepackage{amssymb}
\usepackage{amsfonts}
\usepackage{mathtools}
\usepackage{xspace}
\usepackage{bm}
\usepackage{enumitem}

\usepackage[most]{tcolorbox}

\newtcolorbox{defbox}[1][]
{
	sharp corners,
	colback=cyan!7!white,
	breakable,
	enhanced,
	#1,
}


\newtcolorbox{thmbox}[1][]
{
	sharp corners,
	colback=green!30!white,
	breakable,
	enhanced,
	#1,
}

\newtcolorbox{notebox}[1][]
{
	sharp corners,
	colback=red!5!white,
	breakable,
	enhanced,
	#1,
}



\newtheorem{proposition}{Proposition}
\newtheorem{theorem}{Theorem}
\newtheorem{definition}{Definition}
\newtheorem{lemma}{Lemma}
\newtheorem{claim}{Claim}
\newtheorem{remark}{Remark}
\newtheorem{corollary}{Corollary}
\newtheorem{observation}{Observation}
\newtheorem{example}{Example}

\usepackage{boxedminipage}
\usepackage{xparse}
\usepackage{float}
\usepackage{caption}
\usepackage{ifthen}




%\usepackage{ulem}
\usepackage[
lambda,
operators,
advantage,
sets,
adversary,
landau,
probability,
notions,	
logic,
ff,
mm,
primitives,
events,
complexity,
asymptotics,
keys]{cryptocode}

%comment commands
\newcommand{\arka}[1]{{\color{brown} Arka: #1}}


\renewcommand\labelitemi{--}

%\usepackage{geometry}

\usepackage[pdftex,bookmarks=true,pdfstartview=FitH,colorlinks,linkcolor=Cerulean,filecolor=Cerulean,citecolor=NavyBlue,urlcolor=Cerulean]{hyperref}
\pagestyle{plain}

\newcommand{\bfmatrix}[1]{\mathbf{#1}}
\renewcommand\vec{\mathbf}
\newcommand{\llbrace}{\left\lbrace}
\newcommand{\rrbrace}{\right\rbrace}
\newcommand{\round}[1]{\left\lfloor #1 \right\rceil}
\newcommand{\PR}{\mathbb{P}}
\newcommand\inner[2]{\langle #1, #2 \rangle}
\newcommand{\randomOracle}{\mathcal{O}}
\newcommand{\oracle}[1]{\mathcal{#1}}
\newcommand{\pspace}{\mathsf{PSPACE}}
\newcommand{\iproof}{\mathsf{IP}}
\newcommand{\CRHF}{\mathsf{CRHF}}
\newcommand{\OWP}{\mathsf{OWP}}
\newcommand{\OWF}{\mathsf{OWF}}
\newcommand{\TDF}{\mathsf{TDF}}
\newcommand{\ckt}{\mathsf{C}}
\newcommand{\alice}{\mathsf{Alice}}
\newcommand{\bob}{\mathsf{Bob}}
\newcommand{\eve}{\mathsf{Eve}}
\newcommand{\UOWHF}{\mathsf{UOWHF}}
\newcommand{\dCRHF}{\mathsf{dCRHF}}
\newcommand{\DTIME}{\mathsf{DTIME}}
\newcommand{\DSPACE}{\mathsf{DSPACE}}
\newcommand{\BPTIME}{\mathsf{BPTIME}}
\newcommand{\SZK}{\mathsf{SZK}}
\newcommand{\primitive}[1]{\mathsf{#1}}
\newcommand{\machine}[1]{\mathsf{#1}}
\newcommand{\lang}{\mathsf{L}}

\newcommand{\com}{\mathsf{com}}



\begin{document}
	

\title{\LaTeX definitions for Cryptography Primitives}
\author{}
\date{}
 
\maketitle

Collect definitions of cryptographic primitives so that it is easy to copy and use while writing papers. I will try as far as possible to stick to the macros defined in cryptocode. 



\newpage 
\tableofcontents

\newpage

\chapter{Notation}

\chapter{Probability}
	% !TEX root = ./main.tex

\section{Computational Indistinguishability}

\begin{definition}[Computational Indistinguishability]
	\index{Indistinguishability!Computational}
	Two ensembles $X=\{X_\alpha\}_{\alpha \in S}$ and $Y=\{Y_\alpha\}_{\alpha \in S}$ are said to be computationally indistinguishable, denoted by $X \approx_c Y$, if for every non-uniform $\ppt$ distinguisher $\ddv$, every polynomial $\pp$, all sufficiently large $\secpar$ and every $\alpha \in \bin^{\poly} \cap S$
	\[
		\Big| \prob{\ddv(\secparam,X_\alpha)=1} - \prob{\ddv(\secparam,Y_\alpha)=1} \Big| < \frac{1}{\pp(\secpar)}\enspace,
	\]
	where the probability are taken over the samples of $X_\alpha$, $Y_\alpha$ and coin tosses of $\ddv$.
\end{definition}


\section{Statistical Indistinguishability}

\begin{definition}[Statistical Indistinguishability]
	\index{Indistinguishability!Statistical}
	Two ensembles $X=\{X_\alpha\}_{\alpha \in S}$ and $Y=\{Y_\alpha\}_{\alpha \in S}$ are said to be statistically indistinguishable, denoted by ${X \approx_s Y}$, if for every polynomial $\pp$, all sufficiently large $\secpar$ and every $\alpha \in \bin^{\poly} \cap S$
	\[ 
	 \Delta(X_\alpha,Y_\alpha) < \frac{1}{\pp(\secpar)}\enspace,
	\] 
	where $\Delta(X_\alpha,Y_\alpha)$ corresponds to the statistical distance between $X_\alpha$ and $Y_\alpha$.
\end{definition}



\chapter{Symmetric Key Primitives}
	% !TEX root = ./main.tex

\subsection{Non-interactive Commitment Schemes}
	\label{sec:defn:commitment}
	
	We define below bit commitment schemes
	\begin{definition}[Non-interactive Bit Commitment Schemes]
		A polynomial time computable function: $\com: \bin \times \bin^\secpar \mapsto \bin^{\ell(\secpar)}$ is a bit commitment if it satisfies the properties below:
		\begin{description}
			\item[Binding:] For any $r,r' \in \bin^\secpar, b,b' \in \bin$, if $\com(b;r)=\com(b';r')$ then $b=b'$.
			\item[Computational Hiding:] The following holds:
			\[
				\Big\lbrace \com(0):r \sample \bin^\secpar \Big\rbrace \approx_c \Big\lbrace \com(1;r):r \sample \bin^\secpar \Big\rbrace \enspace.
			\]			
			where computational indistinguishability is with respect to arbitrary non-uniform $\ppt$ distinguisher.
		\end{description}
		
	\end{definition}

\chapter{Public Key Primitives}

\chapter{Proofs}
	% !TEX root = ./main.tex



\chapter{Secure Computation}

\chapter{Obfuscation}





\newpage
\bibliographystyle{alpha}
\bibliography{./cryptobib/abbrev0,./cryptobib/crypto,./extrabib}



\end{document}
