% !TEX root = ./main.tex

\section{One-way Function}

\begin{definition}[One-way Function]
		\index{One-way Function}
		\label{def:strongOWF}
		A function $f:\bin^{*} \mapsto \bin^{*}$ is a one way function if it satisfies the following two conditions: 
		\begin{enumerate}
			\item \emph{ Easy to compute}: There is a $\ppt$ algorithm $C$ s.t. $\forall x \in \{0, 1\}^{*}$,
			\[
				\prob{r \sample \bin^{m}~:~C(x;r) = f(x)} =1.
			\]
			\item \emph{Hard to invert}: For every non-uniform $\ppt$ adversary $\adv$, 
			\[
				\probsublong{x \sample \bin^\secpar, \widetilde{x} \leftarrow \adv(\secparam, f(x))}{f(\widetilde{x})=f(x)} \le \negl
			\]		
		\end{enumerate}
	\end{definition}


\section{Pseudorandom Generators}


\begin{definition}[Psedudorandom Generators]
	\index{Pseudorandom Generators (PRG)}
	A deterministic function $\prg: \bin^\secpar \rightarrow \bin^{p(\secpar)}$ is called a pseudorandom generator (PRG) if:
	\begin{enumerate}
		\item (efficiency): $\prg$ can be computed in polynomial time,
		\item (expansion): $p(\secpar) > \secpar$,
		\item $\llbrace x \sample \bin^\secpar: \prg(x) \rrbrace \approx_c \llbrace U_{p(\secpar)}\rrbrace$ , where $U_{p(\secpar)}$ is the uniform distribution over $p(\secpar)$ bits.
	\end{enumerate}
\end{definition}


\section{Non-interactive Commitment Schemes}
	
	We define below bit commitment schemes
	\begin{definition}[Non-interactive Bit Commitment Schemes]
		\index{Non-interactive Commitment}
		A polynomial time computable function: $\com: \bin \times \bin^\secpar \mapsto \bin^{\ell(\secpar)}$ is a bit commitment if it satisfies the properties below:
		\begin{description}
			\item[Binding:] For any $r,r' \in \bin^\secpar, b,b' \in \bin$, if $\com(b;r)=\com(b';r')$ then $b=b'$.
			\item[Computational Hiding:] The following holds:
			\[
				\Big\lbrace \com(0):r \sample \bin^\secpar \Big\rbrace \approx_c \Big\lbrace \com(1;r):r \sample \bin^\secpar \Big\rbrace \enspace.
			\]			
			where computational indistinguishability is with respect to arbitrary non-uniform $\ppt$ distinguisher.
		\end{description}
		
	\end{definition}


	\section{Signature Scheme}
		
	
	
	\begin{definition}
		\index{Signature Scheme}
		\label{def:signatures}
		An signature scheme consists of three polynomial-time algorithms $(\gen,\sign,\verify)$. 
		\begin{itemize}
			\item[--] $\gen$ is $\ppt$ algorithm that takes as input $\secparam$ and generates a key and verification key. $(\sk,\vk) \leftarrow \gen(\secparam)$.
			\item[--] $\sign$ is a $\ppt$ algorithm that computes the signature on a message $m$.  $\sigma \coloneqq \sign(\sk,m)$.
			\item[--] $\verify$ is a deterministic algorithm verifies the signature using the verification key. $\verify(\vk, m,\sigma)$ returns 0 or 1.
		\end{itemize}
		A  signature scheme that is existentially unforgeable against chosen message attacks if the following hold. 
			\begin{description}
				\item[Correctness] For every message $m \in \mathcal{M}$ (message space),
				\[
					\prob{\verify(\vk,m,\sigma) =1~:~(\vk,\sk) \leftarrow \gen(\secparam), \sigma \leftarrow\sign(\sk,m)} = 1
				\]
				\item[Security] For any $\ppt$ adversary $\adv$
				\[
						\probsublong{\begin{array}{l}
						(\vk,\sk) \leftarrow \gen(\secparam)\\
						(m,\sigma) \leftarrow \adv^{\sign(\sk,\cdot)}(\vk)
					\end{array}}{ \begin{array}{l}
								\adv \text{ did not query }m\\
								\verify(\vk,m,\sigma) =1
						\end{array} 
					} < \negl 
				\]
				where $\adv^{\sign(\sk,\cdot)}$ indicates that $\adv$ has access to an oracle that returns the signature on the queried message $m$.
			\end{description}

	\end{definition}